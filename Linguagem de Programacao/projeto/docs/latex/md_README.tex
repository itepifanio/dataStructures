Observações\+: Esse programa utiliza o pacote boost em sua utilização, certifique-\/se de te-\/lo instalado em sua máquina. Caso o caminho do pacote seja diferente de {\ttfamily -\/\+L/usr/include/boost} então modifique a importação feita no makefile.

Para compilar basta rodar\+: {\ttfamily make}

Para executar basta rodar\+: {\ttfamily bin/exec}

\subsection*{\hyperlink{classCorrida}{Corrida} de Sapos}

Implemente em C++ um programa que simule a uma corrida de sapos. Implemente uma classe chamada \hyperlink{classSapo}{Sapo} contendo\+:


\begin{DoxyItemize}
\item Atributos\+: nome, identificador, distância percorrida, quantidade de pulos dados, quantidade de provas disputadas, vitórias, empates, quantidade total de pulos dados, quantidade.
\item Atributo estático público\+: distância total da corrida
\item Métodos públicos\+:
\begin{DoxyItemize}
\item Construtor(es).
\item getters e setters, quando necessários.
\item pular\+:
\begin{DoxyItemize}
\item incrementa distância percorrida de forma randômica entre 1 e o máximo que o sapo pode saltar
\item Incrementa o número de pulos dados em uma unidade
\end{DoxyItemize}
\end{DoxyItemize}
\item Sinta-\/se a vontade para adicionar algum outro método ou atributos
\end{DoxyItemize}

\subsubsection*{Especificações do projeto}

O código desenvolvido deve seguir as especificações abaixo\+:


\begin{DoxyItemize}
\item O programa deverá ler um arquivo que conterá as informações dos sapos disponíveis para a corrida.
\item O programa deverá ler um arquivo que conterá as informações sobre as pistas disponíveis para a corrida.
\item Ao iniciar o programa, o usuário poderá\+:
\begin{DoxyItemize}
\item Ver estatísticas dos sapos.
\item Ver estatísticas das corridas.
\item Iniciar uma corrida.
\item Criar sapos.
\item Criar corrida.
\end{DoxyItemize}
\item Para iniciar uma corrida\+:
\begin{DoxyItemize}
\item O usuário deverá escolher uma pista de corrida que os sapos irão disputar.
\item É mostrado ao usuário a lista dos sapos que iram participar da corrida com seus nomes e números.
\item O usuário dará o start (pei) da corrida.
\end{DoxyItemize}
\item Durante a corrida\+:
\begin{DoxyItemize}
\item Cada sapo irá pular individualmente, mostrando ao usuário seu nome, numeração e quanto ele pulou, em cada pulo.
\item A medida que um sapo chegar na linha de chegada, ele não deverá mais pular nem emitir mais mensagens na tela.
\item Quando o ultimo sapo terminar a corrida, o programa mostrará o Rank da corrida.
\end{DoxyItemize}
\item Lembre que as operações de criação de sapos e corridas devem salvar os mesmo nos arquivos, bem como as estatísticas dos sapos após as corridas realizadas.
\end{DoxyItemize}

\subsubsection*{Observações}

O código deverá ser devidamente comentado e anotado para dar suporte à geração automática de documentação no formato de páginas Web (H\+T\+ML) utilizando a ferramenta Doxygen. Para maiores informações, você poderá acessar a página do Doxygen na Internet (\href{http://www.doxygen.org/}{\tt http\+://www.\+doxygen.\+org/}).

O código do projeto deve seguir a configuração de pastas e arquivos\+:


\begin{DoxyItemize}
\item /bin – código executável
\item /src – código fonte
\item /docs – documentação
\item makefile
\item R\+E\+A\+D\+ME – arquivo contendo informações sobre\+: configuração, compilação e execução. Também deve conter uma sessão com as informações sobre quais arquivos e as linhas que contêm os Itens Avaliados da tabela 1. 
\end{DoxyItemize}